% MCS Mountain Plot Notebook
\section{Mountain Plots}

%------------------------------------------------------------------------------%
% http://peer.ccsd.cnrs.fr/docs/00/75/39/50/PDF/PEER_stage2_10.1016%252Fj.spl.2011.03.014.pdf
The folded cumulative distribution function for a random variable can be easily obtained by folding down the upper half of the cumulative distribution
function (CDF). It is a simple graphical method for summarising distributions, and has been used for the evaluation of laboratory assays, clinical trials
and quality control (Monti, 1995; Krouwer and Monti, 1995).

%------------------------------------------------------------------------------%

% http://www.ncbi.nlm.nih.gov/pubmed/8547437

A mountain plot (or ``folded empirical cumulative distribution plot") is created by computing a percentile for each ranked difference between a new method and a reference method. 

To get a folded plot, the following transformation is performed for all percentiles above 50: 
percentile = 100 – percentile. These percentiles are then plotted against the differences between 
the two methods (Krouwer \& Monti, 1995). The calculations and plots are simple enough to perform in a spreadsheet. 

The mountain plot is a useful complementary plot to the Bland & Altman plot. 
In particular, the mountain plot offers the following advantages:
%------------------------------------------------------------------------------%
\begin{itemize}
\item It is easier to find the central 95\% of the data, even when the data are not Normally distributed.
\item Different distributions can be compared more easily.
\item Unlike a histogram, the plot shape is not a function of the intervals. 
\item
\item
\end{itemize}

%------------------------------------------------------------------------------%

Compared with the Bland-Altman difference plot, the folded CDF stresses more the median and tails of the difference. If the two assays are ‘unbiased’
98 with each other (Krouwer and Monti, 1995), the median would be close to zero. If the variability between the two assays is large, the width near the
100 bottom of the folded CDF would be large, analogously to a confidence interval.

%------------------------------------------------------------------------------%
%Advantages 
% 1. It is easier to find the central 95\% of the data. 
% 2. It is easier to estimate percentile for large differences (e.g., percentiles greater than 95\%). 
% 3. Unlike a histogram, the plot shape is not a function of the intervals. 
% 4. Comparing different distributions is easier. 
% 5. The plot is easier to interpret than a standard empirical cumulative distribution plot. 

%------------------------------------------------------------------------------%

Bland-Altman and mountain plots each provide complementary perspectives on the data, and the authors recommend both plots.
%------------------------------------------------------------------------------%
