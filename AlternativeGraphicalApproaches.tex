% Identity Plot
% Variations of the Bland Altman Plot
% Survival Plot (Luiz et al)
% Bartko's Ellipse
% Mountain Plot

\documentclass[Chap2bmain.tex]{subfiles}
\begin{document}

\section{The Identity plot} This is a simple regression based approach. It gives the analyst a cursory examination of how well the measurement methods agree. In the case of good agreement, the covariates of the plot accord closely with the $X=Y$ line.

\section{Variations of the BA plot}

In light of some potential pitfalls associated with the conventional BA plot, a series of alternative formulations for the Bland-Altman plot have been proposed.



\section{Survival Agreement Plot (Luiz et al)}
This approach is put forward by \textit{Luiz}. It seeks to extend the agreement evaluation through a graphic approach using step functions' capable of expressing the degree of agreement (or disagreement) as a function of several limits of tolerance.

\begin{itemize}
\item It expresses agreement or disagreement as a function of several
limits of tolerance.
\item Y axis represents the proportion of discordant cases.
\item X axis represents the observed differences.
\end{itemize}

\section{Eksborg's Plot}
\textit{Eksborg} proposes a plot of the relative values found by the two
Methods being compared (Method 1/Method 2) vs the mean of the Method
values.

This approach was discussed as an alternative to the BA approach by 
%%%%%%%%%%%%%%%%%%%%%%%%%%%%%%%%%%%%%%%%%%%%%%%%%%%%%%%%%%%%%%%%%%%%%%%%%
\newpage
\section{Bartko's Ellipse}
As an enhancement on the Bland Altman Plot, \textit{bartko} has
expounded a confidence ellipse for the covariates. \textit{bartko} proposes
a bivariate confidence ellipse as a boundary for dispersion. The stated purpose is to 'amplify dispersion', which presumably is for  the purposes of outlier detection.The orientation of the the ellipse is key to interpreting the results.
\begin{itemize}
 \item The Minor Axis is related to the Variance between-subjects
 \item The Major Axis is related to the Error Mean Square.
\end{itemize}
The ellipse illustrates the size of both relative to each
other. Furthermore, the ellipse provides a visual aid to determining the relationship
between the variance of the means $Var(a_{i})$ and the variance of the differences $Var(d_{i})$.
\begin{itemize}
 \item If $Var(a_{i})$ is greater than $Var(d_{i})$, the orientation of the ellipse is horizontal.
 \item If $Var(a_{i})$ is less than $Var(d_{i})$, the orientation of the ellipse is vertical.
\end{itemize}
The more horizontal the ellipse, the greater the ICC.

\newpage



%----------------------------------------------------------------%
\section{dewitte et al }
Bland ALtman recommend the logarithmic y scale
others prefer the  precent y scale.
generally there is not much difference(except when the data extends over several orders of magnitude)
percent method is recommends becuase the numbers can be read directly from the plot without the need for back transfromation.

\begin{verbatim}
absolute - small range
percentage - medium range
log scale - large range
\end{verbatim}
we observe increasing use of the bland altman plot over the years, from 8% in 1995 to 14% in 1996 and 31%to36% in more recent years.


\end{document} 
